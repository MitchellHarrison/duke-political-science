\documentclass[titlepage, 12pt, leqno]{article}

% -------------------------------------------------- %
% -------------------- PACKAGES -------------------- %
% -------------------------------------------------- %
\usepackage{import}
\usepackage{pdfpages}
\usepackage{mathtools}
\usepackage{transparent}
\usepackage{xcolor}
\usepackage{tcolorbox}
\usepackage{amsmath}
\usepackage{amssymb}
\usepackage{parskip}
\usepackage{bbm}
\usepackage[margin = 1in]{geometry}
\tcbuselibrary{breakable}
\tcbset{breakable = true}


% -------------------------------------------------- %
% -------------- CUSTOM ENVIRONMENTS --------------- %
% -------------------------------------------------- %
\newtcolorbox{note}{colback=black!5!white,
                          colframe=black!55!white,
                          fonttitle=\bfseries,title=Note}

\newtcolorbox{ex}{colback=blue!5!white,
                          colframe=blue!55!white,
                          fonttitle=\bfseries,title=Example}

\newtcolorbox{definition}{colback=red!5!white,
                          colframe=red!55!white,
                          fonttitle=\bfseries,title=Definition}


% -------------------------------------------------- %
% ------------------- COMMANDS --------------------- %
% -------------------------------------------------- %
% Brackets, braces, etc. 
\newcommand{\abs}[1]{\lvert #1 \rvert}
\newcommand{\bigabs}[1]{\Bigl \lvert #1 \Bigr \rvert}
\newcommand{\bigbracket}[1]{\Bigl [ #1 \Bigr ]}
\newcommand{\bigparen}[1]{\Bigl ( #1 \Bigr )}
\newcommand{\ceil}[1]{\lceil #1 \rceil}
\newcommand{\floor}[1]{\lfloor #1 \rfloor}
\newcommand{\norm}[1]{\| #1 \|}
\newcommand{\bignorm}[1]{\Bigl \| #1 \Bigr \| #1}
\newcommand{\inner}[1]{\langle #1 \rangle}
\newcommand{\set}[1]{{ #1 }}


% -------------------------------------------------- %
% -------------------- SETUP ----------------------- %
% -------------------------------------------------- %
\title{\Huge{Lecture 2 - Introduction to Modelling}}
\author{\large{Mitch Harrison}}
\date{\today}   
\begin{document}
\setlength{\parskip}{1\baselineskip}
\setlength{\parindent}{15pt}
\maketitle
\tableofcontents
\newpage


% -------------------------------------------------- %
% --------------------- BODY ----------------------- %
% -------------------------------------------------- %
\section{Thinking for Research}
\subsection{Forming research questions}
Each research question has to give rise to an experiment/model with a 
\textit{single} response variable and as few dependent variables as possible
(ideally one). Specifying your dependent variables allows you to systematically
vary that one variable to find its impact on the response. In general, extra
parameters tend to appear as models are constructed. For example, if we are
simulating population movements based on wealth, other factors (ease of movement,
border conditions, etc.) naturally arise from the nature of the simulation.

\pagebreak
\section{Simulation Examples}
\subsection{Sugar creatures}
In this famous example, dots live on a 2-dimensional plane. "Sugar" is placed on
the plane, and dots pursue the sugar. Sugar increases their life expectancy, and 
until they find sugar, they move at random. Lifespans, movement speed, amount and
position of sugar, and other factors are variable. 

\subsection{Vote-winning}
Similar to the sugar creatures, one can map political belief on a 2-dimensional
cartesian plane (the "Political Compass"). Voters form a shape on that plane 
similar to sugar. A party/candidate sits on the same plane, and wins the
election if they are near a larger mass of the area of voters. Naturally, 
both parties will gravitate towards the center of mass of the voter region. 

\begin{note}
    In these models, voters are not \textbf{actors}. They are an element of the
    environment, while the \textit{parties} are actors.
\end{note}

In more complex models, the voter region shifts over time. In even more complex 
models, parties have some control over the movement over the voting mass (via
media, campaigns, advertising, etc.). 

\begin{note}
    In these more complex models, voters \textit{and} parties are actors, since
    both change in response to a stimulation (in this case, from each other).
\end{note}

\end{document}
