\documentclass[titlepage, 12pt, leqno]{article}

% -------------------------------------------------- %
% -------------------- PACKAGES -------------------- %
% -------------------------------------------------- %
\usepackage{import}
\usepackage{pdfpages}
\usepackage{mathtools}
\usepackage{transparent}
\usepackage{xcolor}
\usepackage{tcolorbox}
\usepackage{amsmath}
\usepackage{amssymb}
\usepackage{enumitem}
\usepackage{parskip}
\usepackage{bbm}
\usepackage[margin = 1in]{geometry}
\tcbuselibrary{breakable}
\tcbset{breakable = true}


% -------------------------------------------------- %
% -------------- CUSTOM ENVIRONMENTS --------------- %
% -------------------------------------------------- %
\newtcolorbox{note}{colback=black!5!white,
                          colframe=black!55!white,
                          fonttitle=\bfseries,title=Note}

\newtcolorbox{ex}{colback=blue!5!white,
                          colframe=blue!55!white,
                          fonttitle=\bfseries,title=Example}

\newtcolorbox{definition}{colback=red!5!white,
                          colframe=red!55!white,
                          fonttitle=\bfseries,title=Definition}


% -------------------------------------------------- %
% ------------------- COMMANDS --------------------- %
% -------------------------------------------------- %
% Brackets, braces, etc. 
\newcommand{\abs}[1]{\lvert #1 \rvert}
\newcommand{\bigabs}[1]{\Bigl \lvert #1 \Bigr \rvert}
\newcommand{\bigbracket}[1]{\Bigl [ #1 \Bigr ]}
\newcommand{\bigparen}[1]{\Bigl ( #1 \Bigr )}
\newcommand{\ceil}[1]{\lceil #1 \rceil}
\newcommand{\floor}[1]{\lfloor #1 \rfloor}
\newcommand{\norm}[1]{\| #1 \|}
\newcommand{\bignorm}[1]{\Bigl \| #1 \Bigr \| #1}
\newcommand{\inner}[1]{\langle #1 \rangle}
\newcommand{\set}[1]{{ #1 }}


% -------------------------------------------------- %
% -------------------- SETUP ----------------------- %
% -------------------------------------------------- %
\title{\Huge{Lecture 5 - Deductive Computational Modelling}}
\author{\large{Mitch Harrison}}
\date{\today}   
\begin{document}
\setlength{\parskip}{1\baselineskip}
\setlength{\parindent}{15pt}
\maketitle
\tableofcontents
\newpage


% -------------------------------------------------- %
% --------------------- BODY ----------------------- %
% -------------------------------------------------- %
\section{Deductive Modelling}

\subsection{Periods, Steady State, and Batch Runs}

\begin{definition}
    A \textbf{period} is a single iteration of a single run of a simulation.
\end{definition}

\begin{definition}
    A \textbf{steady state} occurs in a simulation when the system "settles" into
    a state that ceases to evolve (at least ceases to evolve rapidly).
\end{definition}

\begin{definition}
    \textbf{Batch runs} aggregate many sample paths to obtain mean behavior for 
    the steady state at particular parameter values.
\end{definition}

Examining many sample paths can be helpful to determine when the process settles
down. Of course, it is best to know that the process will settle before doing 
this. If it doesn't, there is no steady state to analyze. This is where
\textit{ergodicity} comes in.

One can then discard the \textit{burn-in period} if one is interested only in 
steady-state behavior, but is averaging across periods. This eliminates early
random behavior. The number of period \textbf{must} be specified as a variable in
a batch run.

\subsection{Comparative Statistics}

There are several ways to derive comparative statistics from a simulation. These 
are needed because fully computing derivatives of steady states with respect to
parameters is basically impossible with more than a handful of parameters.
Perhaps the simplest involves randomly sampling the parameter space to derive a 
simulation data set, and then regressing the steady-state variables on the
parameters.With enough sets of parameter values, or few enough parameters, one 
can obtain a fair representation of the linear, directional effects of the 
parameters. 

An alternative, more careful substitute method is to take data at
regularized points in the parameter space and calculate numerical
approximations to the partial derivatives at each of these points. If
they all point in the same direction, one can be pretty sure of the
sign of the comparative static.

There are several problems with this approach:
\begin{enumerate}[itemsep=-2pt]
    \item It ignores the likelihood of interactions between parameters.
    \item It ignored the likelihood of non-linearity and non-monotonicity.
    \item It likely indersamples in regions of rapid change, should such exist,
        potentially missing critical behavior.
\end{enumerate}

\subsection{Sequential parameter sweeping}

An alternative method eschews sampling in favor of detailed \textit{parameter
sweeps}. This approach produces fine-grainedcomparative statistics for a given
variable, for any particular values of the other variables. It helps to ensure
that the computational model is theoretically tractable.

However, to perform these sweeps, one has to substantially limit the parameter
space of the model, which is not always feasible. This is due to the
computational workload required to perform these sweeps. Optimizing or
parallelizing code can speed up this work with comparatively little impact on the
final product. However, it is often easier to simplify the model itself.

One way to minimize this computational issue is to build the model in stages,
using theoretical justifications. We begin with a simple model with 1-3 parameters
and analyze it fully by numerically computing comparative statistics. Then, we see
if we can separate this parameter space into regions that have similar comparative
statistics. If this is successful, we can label each region as theoretically
distinct. We can use this technique for larger parameter spaces through its
iterative application.

\end{document}
