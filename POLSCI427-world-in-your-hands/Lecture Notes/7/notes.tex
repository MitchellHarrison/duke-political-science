\documentclass[titlepage, 12pt, leqno]{article}

% -------------------------------------------------- %
% -------------------- PACKAGES -------------------- %
% -------------------------------------------------- %
\usepackage{import}
\usepackage{pdfpages}
\usepackage{mathtools}
\usepackage{transparent}
\usepackage{enumitem}
\usepackage{xcolor}
\usepackage{tcolorbox}
\usepackage{amsmath}
\usepackage{amssymb}
\usepackage{parskip}
\usepackage{bbm}
\usepackage[margin = 1in]{geometry}
\tcbuselibrary{breakable}
\tcbset{breakable = true}


% -------------------------------------------------- %
% -------------- CUSTOM ENVIRONMENTS --------------- %
% -------------------------------------------------- %
\newtcolorbox{note}{colback=black!5!white,
                          colframe=black!55!white,
                          fonttitle=\bfseries,title=Note}

\newtcolorbox{ex}{colback=blue!5!white,
                          colframe=blue!55!white,
                          fonttitle=\bfseries,title=Example}

\newtcolorbox{definition}{colback=red!5!white,
                          colframe=red!55!white,
                          fonttitle=\bfseries,title=Definition}


% -------------------------------------------------- %
% ------------------- COMMANDS --------------------- %
% -------------------------------------------------- %
% Brackets, braces, etc. 
\newcommand{\abs}[1]{\lvert #1 \rvert}
\newcommand{\bigabs}[1]{\Bigl \lvert #1 \Bigr \rvert}
\newcommand{\bigbracket}[1]{\Bigl [ #1 \Bigr ]}
\newcommand{\bigparen}[1]{\Bigl ( #1 \Bigr )}
\newcommand{\ceil}[1]{\lceil #1 \rceil}
\newcommand{\floor}[1]{\lfloor #1 \rfloor}
\newcommand{\norm}[1]{\| #1 \|}
\newcommand{\bignorm}[1]{\Bigl \| #1 \Bigr \| #1}
\newcommand{\inner}[1]{\langle #1 \rangle}
\newcommand{\set}[1]{{ #1 }}


% -------------------------------------------------- %
% -------------------- SETUP ----------------------- %
% -------------------------------------------------- %
\title{\Huge{Lecture 7 - Social Networks cont'd}}
\author{\large{Mitch Harrison}}
\date{\today}   
\begin{document}
\setlength{\parskip}{1\baselineskip}
\setlength{\parindent}{15pt}
\maketitle
\tableofcontents
\newpage


% -------------------------------------------------- %
% --------------------- BODY ----------------------- %
% -------------------------------------------------- %
\section{Social Networks}

\begin{definition}
    A \textbf{social network} (either in person or online) is a network of social
    connections among agents. This could be an immediate real-life friend
    group, Facebook groups, etc.
\end{definition}

Recall that there are two types of \textit{repression}: cutting ties and removal.
In cutting of ties, a network is broken up to sever connection between agents to
force/prevent action. In removal, individuals are removed from the network 
entirely, which could be anything from gag orders to political killings.

There are other considerations to be made when removing agents from a network.
There are two possible strategies: random selection or targeted selection. Random
selection does what it says on the tin: chooses people uniformly at random to
remove from the network. Targeted selection seeks to find the most influential
agents in a social network and removes them in an attempt to maximize value for
each individual removal.

\subsection{Emotional responses}
Emotional response to the removal of members of a social network (friends, family,
etc.) can be modelled as well. For example, if a user feels:
\begin{itemize}
    \item anger: they may be more likely to act in rebellion against the remover
    \item fear: the may be \textit{less} likely to act in rebellion.
\end{itemize}

\pagebreak
\section{Group Project}
\textbf{Question:} Which factors increase aggregate voter turnout per capita?
\subsection{Rules}
\begin{itemize}
    \item If $X$\% of an agent's social network is voting for the same candidate,
        they will turn out to vote.
    \item If an agent consumes $Y$ hours of political news per week, they will
        turn out.
    \item If an agent works a full-time work day on election day, they \textbf{can
        not} turn out to vote, regardless of other rules.
    \item $N$\% of agents will turn out no matter what.
    \item For every $M$\% that a candidate is leading, some agents will stop 
        turning out because they are confident a candidate will win without them.
    \item \texttt{True/False} if an agent is economically "comfortable", they
        will turn out
\end{itemize}

\subsection{Inputs}
\begin{itemize}
    \item starting polling numbers ($M$\% chance than candidate $A$ wins)
    \item percent $N$ of agents that will turn out no matter what
    \item alignment to specific candidates
    \item marketing budget
    \item accessability to voting (polling distance, registration, etc.)
    \item how far apart candidates are from one another
    \item number of candidates
    \item number of major parties
    \item approval rating of incumbent
\end{itemize}

\subsection{Outputs}
\begin{itemize}
    \item voter turnout per capita
\end{itemize}

\end{document}
