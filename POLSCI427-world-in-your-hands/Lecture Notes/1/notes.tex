\documentclass[titlepage, 12pt, leqno]{article}

% -------------------------------------------------- %
% -------------------- PACKAGES -------------------- %
% -------------------------------------------------- %
\usepackage{import}
\usepackage{pdfpages}
\usepackage{mathtools}
\usepackage{transparent}
\usepackage{xcolor}
\usepackage{tcolorbox}
\usepackage{amsmath}
\usepackage{amssymb}
\usepackage{parskip}
\usepackage{bbm}
\usepackage[margin = 1in]{geometry}
\tcbuselibrary{breakable}
\tcbset{breakable = true}


% -------------------------------------------------- %
% -------------- CUSTOM ENVIRONMENTS --------------- %
% -------------------------------------------------- %
\newtcolorbox{note}{colback=black!5!white,
                          colframe=black!55!white,
                          fonttitle=\bfseries,title=Note}

\newtcolorbox{ex}{colback=blue!5!white,
                          colframe=blue!55!white,
                          fonttitle=\bfseries,title=Example}

\newtcolorbox{definition}{colback=red!5!white,
                          colframe=red!55!white,
                          fonttitle=\bfseries,title=Definition}


% -------------------------------------------------- %
% ------------------- COMMANDS --------------------- %
% -------------------------------------------------- %
% Brackets, braces, etc. 
\newcommand{\abs}[1]{\lvert #1 \rvert}
\newcommand{\bigabs}[1]{\Bigl \lvert #1 \Bigr \rvert}
\newcommand{\bigbracket}[1]{\Bigl [ #1 \Bigr ]}
\newcommand{\bigparen}[1]{\Bigl ( #1 \Bigr )}
\newcommand{\ceil}[1]{\lceil #1 \rceil}
\newcommand{\floor}[1]{\lfloor #1 \rfloor}
\newcommand{\norm}[1]{\| #1 \|}
\newcommand{\bignorm}[1]{\Bigl \| #1 \Bigr \| #1}
\newcommand{\inner}[1]{\langle #1 \rangle}
\newcommand{\set}[1]{{ #1 }}


% -------------------------------------------------- %
% -------------------- SETUP ----------------------- %
% -------------------------------------------------- %
\title{\Huge{Lecture 1 - Introductions}}
\author{\large{Mitch Harrison}}
\date{\today}   
\begin{document}
\setlength{\parskip}{1\baselineskip}
\setlength{\parindent}{15pt}
\maketitle
\tableofcontents
\newpage


% -------------------------------------------------- %
% --------------------- BODY ----------------------- %
% -------------------------------------------------- %
\section{Course Information}
\subsection{Overview}
This is a pseudo-Capstone course, with only roughly $\frac{1}{3}$ of the course
comprising of lectures and introductions to Python and the remaining will be
guidance for the final project. Research questions will be formulated with the
group.

\begin{note}
    \textbf{Example research topics:}
    \begin{itemize}
        \item climate and conflict/migration
        \item election outcomes
        \item policy outcomes
    \end{itemize}
\end{note}


The middle portion of the course will be dedicated to group project work, and
the end will be another research project without working with the group. There
will be two short problem sets early in the course. One will involve Python
programming and the other will be progress towards research questions.

\subsection{Course substance}
Game theory is prone to errors caused by formality. In the real world, human
actors don't always make the optimal decision. Instead of formal mathematical
systems, we turn to programatic simulations that are programmed to include 
approximations of these human heuristics. This course attempts to bridge the gap
between these practices by formalizing these computational models.

\begin{definition}
    \begin{itemize}
        \item \textbf{Actors} can be nations, political parties, individuals, or
            others.
        \item \textbf{Behavioral Rules} are ways that actors respond to changes 
            in their environment.
    \end{itemize}
\end{definition}

Research projects will require any number of actors and behavioral rules, but
there will be \textit{exactly} one dependent variable and \textit{at least} one
independent variable that changes as the simulation progresses. We will minimize
the complexity of our simulations for the sake of explanation and ability to
answer the \textit{why} questions that we are exploring.

\end{document}
