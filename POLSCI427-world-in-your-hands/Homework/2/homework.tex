\documentclass[titlepage, 12pt, leqno]{article}

% -------------------------------------------------- %
% -------------------- PACKAGES -------------------- %
% -------------------------------------------------- %
\usepackage{import}
\usepackage{pdfpages}
\usepackage{mathtools}
\usepackage{transparent}
\usepackage{enumitem}
\usepackage{xcolor}
\usepackage{tcolorbox}
\usepackage{amsmath}
\usepackage{amssymb}
\usepackage{parskip}
\usepackage{bbm}
\usepackage[margin = 1in]{geometry}
\tcbuselibrary{breakable}
\tcbset{breakable = true}


% -------------------------------------------------- %
% -------------- CUSTOM ENVIRONMENTS --------------- %
% -------------------------------------------------- %
\newtcolorbox{note}{colback=black!5!white,
                          colframe=black!55!white,
                          fonttitle=\bfseries,title=Note}

\newtcolorbox{ex}{colback=blue!5!white,
                          colframe=blue!55!white,
                          fonttitle=\bfseries,title=Example}

\newtcolorbox{definition}{colback=red!5!white,
                          colframe=red!55!white,
                          fonttitle=\bfseries,title=Definition}


% -------------------------------------------------- %
% ------------------- COMMANDS --------------------- %
% -------------------------------------------------- %
% Brackets, braces, etc. 
\newcommand{\abs}[1]{\lvert #1 \rvert}
\newcommand{\bigabs}[1]{\Bigl \lvert #1 \Bigr \rvert}
\newcommand{\bigbracket}[1]{\Bigl [ #1 \Bigr ]}
\newcommand{\bigparen}[1]{\Bigl ( #1 \Bigr )}
\newcommand{\ceil}[1]{\lceil #1 \rceil}
\newcommand{\floor}[1]{\lfloor #1 \rfloor}
\newcommand{\norm}[1]{\| #1 \|}
\newcommand{\bignorm}[1]{\Bigl \| #1 \Bigr \| #1}
\newcommand{\inner}[1]{\langle #1 \rangle}
\newcommand{\set}[1]{{ #1 }}


% -------------------------------------------------- %
% -------------------- SETUP ----------------------- %
% -------------------------------------------------- %
\title{\Huge{Homework 2}}
\author{\large{Mitch Harrison}}
\date{\today}   
\begin{document}
\setlength{\parskip}{1\baselineskip}
\setlength{\parindent}{15pt}
\maketitle
\newpage


% -------------------------------------------------- %
% --------------------- BODY ----------------------- %
% -------------------------------------------------- %
\section{Question 1}

\begin{ex}
    Return to the simulation of Schelling’s segregation model that we considered 
    in class. Use the online tool to derive at least three empirical hypotheses 
    from it. An empirical hypothesis can be generated by observing how the 
    model’s outputs (which could be anything from the final level of segregation
    to the time the model takes to settle down) respond to changes in the 
    parameters of the model (here the parameters are things like the \% similar
    or the \% empty, all of which you can vary at the bottom of the web
    page). Discuss each hypothesis in a sentence or two.
\end{ex}

\begin{enumerate}
    \item 100\% racist population cannot be satisfied in under 1500 iterations
        unless there is $>80$\% emptiness, if the racial populations are equal in
        number. This demonstrates that satisfaction is effectively impossible when
        two large racial groups are in conflict in an area.
    \item Even with as little as 10\% of housing being vacant, a population can
        support preferences of at least 75\% racial makeup of neighbors and still
        reach a stable equilibrium. Strong racial segregation is possible, even in
        situations of comparatively low housing availability. This applies only
        to populations where the two races occur in equal proportion.
    \item Having one race in larger numbers (as high as 85:15 ratio) can achieve
        satisfaction above 95\% even with 1\% housing vacancy. Eventually, the
        minority racial group segregated into a single mass. This could be how
        smaller racial communities (e.g. "Chinatown") form in larger cities.
\end{enumerate}

\pagebreak
\section{Question 2}

\begin{ex}
    In a single paragraph, offer a first cut at an idea you might want to explore 
    via a simulation model such as the ones we’ve discussed in class. This should
    be an idea you’re thinking about using for your individual project during the
    last third of the course.
\end{ex}

I am interested in simulating the polarizing effect of political misinformation in
the media. I would model a social network using a directed graph in which some
nodes (that I will call \textit{hubs}) have many more connections to others. These
hubs will represent media sources. Individuals (i.e. non-hub nodes) and hubs will
be one of two colors representing a simplified two-party media structure.
Individuals will also share social connections. Edges (i.e. connections between
nodes) will be weighted to represent the comparative trust an individual places in
their social circle or preferred media. Some hubs will occassionally spread a
piece of misinformation, causing various changes to the weights of the edges
connecting them to individuals. I expect to see that media outlets that output
more misinformation will form a smaller but much more loyal comsumer base than
non-misinformation outlets.

Social connections between individuals will form or break based on their
comparative weights of media trust and party, representing a social break caused
by increased polarization. I expect misinformation sources to cause an echo
chamber that is socially disconnected from outsiders.
 
\pagebreak
\section{Question 3}

\begin{ex}
    For this idea, do the following:
    \begin{enumerate}
        \item Describe the agents in your model. Discuss all their associated
            parameters. Write code that could be used to initialize all the 
            parameters.
        \item Discuss what your agents would do in your model. Write pseudo-code
            (i.e., just describe the steps verbally) that lists the steps the 
            agents would take to accomplish what you want the agents to do.
        \item Discuss what output variables you would like to measure in your 
            model. Write code that would be used to initialize all output 
            measures and pseudo-code that would compute the relevant measures.
    \end{enumerate}
\end{ex}

\begin{enumerate}
    \item Agents in my model will be nodes in my graph. Each will have a series of
        edges connecting them to either a hub (which is also just a node) or
        another individual. Each node (individual \textit{or} hub) will be 
        colored in one of two colors to represent the political leanings of that
        node. Below is example code using a \texttt{Node} object, which 
        simplifies the demonstration of parameters but will likely not be used
        in the final project (where nodes will be represented by NumPy arrays).
        \begin{verbatim}
        individual = Node()
        individual_2 = Node()
        individual_3 = Node()
        hub = Node()

        individual.color = "red"
        individual.neighbors = [individual_2, individual_3, hub]
        individual.edge_weights = [0.5, 1.5, 4]
        \end{verbatim}

    \item Primarily, the changes in each iteration will be to the weights of the
        edges connected to each node. New nodes can also form, while others can
        break. Thus:

        \begin{verbatim}
        for each iteration:
            - edge weights to hubs slightly increase
            - partisan nodes shrink connections to opposite color
            - if hubs release misinformation with some probability p
                - connected nodes change trust levels of that hub
                - inter-node relationships change
            - connection strength is updated between nodes
        \end{verbatim}

    \item I am intersted in multiple output variables. First, I am interested in
        the absolute number of edges connected to each hub after the simulation
        concludes. Second, I am interested in the mean weight of edges connecting
        nodes (i.e. do highly-trusted "echo-chambers" form), and third, I am 
        interested in the absolute number of edges connected to individuals,
        specifically highly-partisan ones. Some example code is included below
        (although simplified).

        \begin{verbatim}
        node_id = 1 # pick a node in the graph
        num_edges = 0
        sum_edge_weight = 0

        for n in my_graph.neighbord(node_id):
            weight = my_graph[node_id][n]["weight"]
            sum_edge_weight = 0
            num_edges += 1

        mean_edge_weight = sum_edge_weight / num_edges
        \end{verbatim}

        This will calculate number of edges and mean edge weight for every node.
        Adding a conditional will help track hub nodes separately, but the code 
        will look effectively identital. 

        \begin{note}
            Here, a \texttt{node} is a NetworkX \texttt{Node} object. NetworkX is
            a Python package for building and analyzing graph networks, and will
            be the backbone of my project.
        \end{note}
\end{enumerate}

\end{document}
